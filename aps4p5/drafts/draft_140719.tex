%\documentclass[prd, twocolumn, nofootinbib, floatfix]{revtex4}
%\documentclass[useAMS,usenatbib]{mn2e}
\documentclass[useAMS,usenatbib]{aastex}



\usepackage{amsmath}
\usepackage{amsbsy}

\topmargin0.0cm
\textheight8.5in


\input epsf
\usepackage{amsmath,amssymb,subfigure}
\usepackage{graphicx}
\usepackage{epsfig}
\usepackage{color}
%\usepackage{ulem}
%\usepackage{epstopdf}

\renewcommand{\topfraction}{0.95}
\renewcommand{\bottomfraction}{0.95}

\newcommand{\be}{\begin{equation}}
\newcommand{\ee}{\end{equation}}
\newcommand{\gscr}{{\mathcal{G}}}
\newcommand{\vscr}{{\mathcal{V}}}

\newcommand{\highlight}[1]{\textcolor{blue}{#1}}

\newcommand{\beq}{\begin{equation}}
\newcommand{\eeq}{\end{equation}}
\newcommand{\beqa}{\begin{eqnarray}}
\newcommand{\eeqa}{\end{eqnarray}}
\newcommand{\om}{\Omega_m}
\newcommand{\omw}{\Omega_w}
\newcommand{\lam}{\Lambda} 
\newcommand{\dcc}{\Delta_m\chi^2} 
\newcommand{\gs}{\gtrsim} 
\newcommand{\ls}{\mathrel{\raise0.27ex\hbox{$<$}\kern-0.70em \lower0.71ex\hbox{{
$\scriptstyle \sim$}}}}
\newcommand{\like}{\chi^2}
\newcommand{\APS}{APS }


%%%%%%%%%%%%%%%%%%%%%%%%%%%%%%%%%%%%%%%%%%%%%%%%%%%%%%%%%%%%
%%%%%%%%%%%%%%%%%%%%%%%%%%%%%%%%%%%%%%%%%%%%%%%%%%%%%%%%%%%%
%%%%%%%%%%%%%%%%%%%%%%%%%%%%%%%%%%%%%%%%%%%%%%%%%%%%%%%%%%%%


\title
{An Efficient Parameter Space Search as an Alternative to Markov Chain Monte Carlo} 
\author
{Scott F.\ Daniel$^1$, Andrew J.\ Connolly$^1$, and Jeff Schneider$^2$\\
$^1$Department of Astronomy, University of Washington, Seattle, WA\\ 
$^2$Machine Learning Department, Carnegie Mellon University, Pittsburgh, PA
}
\begin{document} 


%\pagerange{\pageref{firstpage}--\pageref{lastpage}}

\label{firstpage}

\date{\today}

\maketitle 

%%%%%%%%%%%%%%%%%%%%%%%%%%%%%%%%%%%%%%%%%%%%%%%%%%%%%%%%%%%%
%%%%%%%%%%%%%%%%%%%%%%%%%%%%%%%%%%%%%%%%%%%%%%%%%%%%%%%%%%%%
\begin{abstract} 
We consider the problem of inferring constraints on a high-dimensional parameter
space with a computationally expensive likelihood function.  
Markov chain Monte Carlo (MCMC) methods offer significant improvements in
efficiency over grid-based searches and are easy to implement in a wide range of
cases.  However, MCMCs offer 
few guarantees that all of the interesting regions of 
parameter space are explored.
We propose a
machine learning algorithm that improves upon the performance of MCMC
by intelligently targeting likelihood evaluations so as to quickly and
accurately characterize the likelihood surface in both low- and high-likelihood
regions.  
We compare our algorithm to MCMC on toy examples and the 7-year WMAP
cosmic microwave background data release.  
Our algorithm finds comparable parameter
constraints to MCMC comparably quickly and with greater
certainty that all of the interesting regions of parameter space have been
explored.
\end{abstract} 

\section{Introduction}
\label{sec:intro}

We consider the following problem.  A researcher wants to constrain a theory
described by some number $N_p$ of tunable parameters, which we will represent by
the $N_p$-dimensional vector $\vec{\theta}$.  The available data measure some
function $f$ which is an implicit function of $\vec{\theta}$.  The researcher
is able to calculate some statistic (e.g. $\chi^2$) quantifying the fit of a 
model $f(\vec{\theta})$ to the data and wishes to find a confidence limit
defined by some rule, e.g $\chi^2\le\chi^2_\text{lim}$.  This is a general
phrasing of a Frequentist confidence limit as discussed in Appendix
\ref{sec:frequentism}.  We discuss alternatives for determining
$\chi^2_\text{lim}$ in Section \ref{sec:chi} below.

If the calculation going from $\vec{\theta}\rightarrow\chi^2$ is rapid, this
problem is trivial.  One generates a fine grid of points in the 
$N_p$-dimensional parameter space, evaluates $\chi^2$ at each of these points, 
and defines the confidence limit as only those points which satisfy the rule.  
Often, however, the calcuation $\vec{\theta}\rightarrow\chi^2$ is not rapid, or $N_p$
is large enough that such a grid-based method will take a prohibitively long
time.  In that case, one must make intelligent choices regarding which points in
parameter space are actually interesting enough to warrant
 $\vec{\theta}\rightarrow\chi^2$ evaluation.

Markov Chain Monte Carlo (MCMC) methods propose to find the desired
limits by drawing random samples from the parameter space which are drawn
according to a probability density (the Bayesian posterior) which can be
interpreted as the probability that some value of $\vec{\theta}$ represents the
true value of $\vec{\theta}$ realized by the physical process underlying
the data.  
With enough of these samples, one can integrate this
probability density and find regions which contain some set amount of the total
probability.  In this case, ``credible limits'' are phrased as target values
for the probability integral, i.e. 
$0.68$ for ``1-$\sigma$'' limits, $0.95$ for ``2-$\sigma$'' limits
and so on \cite{mcmc}.

We propose an alternative solution, using methods from machine learning
(to wit, Gaussian processes, though the algorithm admits many possible drivers; see
Section \ref{sec:gp})
to find the desired confidence limit by directly searching the
parameter space of $\vec{\theta}$ for points which lie on the boundary of the
desired confidence limit $\chi^2\le\chi^2_\text{lim}$.  
We refer to this algorithm as Active Parameter Searching (APS).
We present the algorithm in detail in Sections \ref{sec:algorithm} and
\ref{sec:gp}.
As it searches, APS uses
the knowledge it has already learned about $\vec{\theta}\rightarrow\chi^2$ 
to improve its search according to a metric that rewards the
algorithm both for finding points on the confidence limit and for sampling
points from previously unexplored regions of parameter space.  We find that this
behavior makes APS more robust against multi-modal $\chi^2$ functions (see
Section \ref{sec:toy}) without
sacrificing the speed of convergence 
associated with traditional MCMC methods (see Section \ref{sec:wmap}).
We show that APS is capable of yielding both Frequentist confidence
limits (which we define in Appendix \ref{sec:frequentism}) as well as more traditional
Bayesian credible limits on the space of $\vec{\theta}$.


\section{The APS Algorithm}
\label{sec:algorithm}

The APS algorithm was originally presented by Bryan (2007) as

\begin{itemize}
\item(1A) Generate some initial number $N_s$ of randomly-distributed points in 
parameter space.  Find the values of $\chi^2$ corresponding to these points.
\\
\item(2A) Generate some number $N_c$ of candidate points.  
For each of these points, find the $N_g$ nearest points already sampled
by APS. Use these nearest neighbor points and their corresponding $\chi^2$ values
to predict the value of $\chi^2$ at each of these points.
 We will signify this prediction by $\mu$.  Calculate some value $\sigma$
 quantifying confidence in $\mu$ as a prediction of $\chi^2$.  Low values of
 $\sigma$ should correspond to confident predictions.
 \\
\item(3A) Choose the candidate point which maximizes the statistic
\begin{equation}
\label{eqn:sstat}
S=\sigma-|\mu-\chi^2_\text{lim}|
\end{equation}
and evaluate $\vec{\theta}\rightarrow\chi^2$ at that point.  Add this point and
its $\chi^2$ value to the list of sampled points begun in (1A). 
Return to (2A) and repeat until convergence.
\\
\end{itemize}
The confidence limit reported by APS is the list of all points found
for which $\chi^2\le\chi^2_\text{lim}$.  
These points, when plotted in one- or two-dimensional slices of parameter
space, ought to sketch out a region which contains the true value of $\vec{\theta}$
with the desired confidence (as defined in Appendix \ref{sec:frequentism}).
This is in contrast to the credible
limits of MCMC, which represent an integral over the sampled points containing some
fraction of the total posterior probability.

The $|\mu-\chi^2_\text{lim}|$ term in equation (\ref{eqn:sstat}) provides an
incentive for APS to choose points that it believes will lie on the boundary of
the confidence limit.  
The $\sigma$ term provides an incentive for APS to explore unknown regions of
parameter space, making the algorithm robust against multimodal $\chi^2$
functions.  

In the present work, we introduce modifications and expansions to this basic algorithm as
described below.

\subsection{Determining $\chi^2_\text{lim}$}
\label{sec:chi}

As discussed above, APS draws confidence limits by mapping
iso-$\chi^2$ contours of $\chi^2=\chi^2_\text{lim}$ in parameter space.
Thus, the user is presented with the problem of determining an
appropriate $\chi^2_\text{lim}$.  APS admits two possible solutions.
The user can fiat a definite value of $\chi^2_\text{lim}$
based on the characteristics of the data at hand,
or $\chi^2_\text{lim}$ can be set adaptively, based on APS'
knowledge of $\vec{\theta}\rightarrow\chi^2$.

The first possibility, a user-defined $\chi^2_\text{lim}$ is based
on the theory of Frequentist confidence intervals, which we briefly discuss
here and consider in much greater detail in Appendix \ref{sec:frequentism}.
If a data set is comprised of $N_d$ Gaussian-distributed data points
with a known covariance structure, then 
\begin{equation}
\chi^2\equiv\sum_{i,j}^{N_d}(d_i-f_i(\vec{\theta}))C^{-1}_{i,j}
(d_j-f_j(\vec{\theta}))
\end{equation}
will be distributed according to a $\chi^2$-distribution with $N_d$ degrees
of freedom.  This distribution can be integrated to determine the value
of $\chi^2$ containing $(1-\alpha)\%$ of the total probability.  This limiting
value will be the user-defined $\chi^2_\text{lim}$.

If the user does not feel comfortable settig $\chi^2_\text{lim}$
a priori, APS $\chi^2_\text{lim}$ can learn $\chi^2_\text{lim}$ by
defining it as $\chi^2_\text{min}+\Delta\chi^2$.  Here, $\chi^2_\text{min}$
is the minimum value of $\chi^2$ discovered by APS.  
$\Delta \chi^2$ is set
based on the properties of the $N_p$-dimensional parameter space being explored.
To wit, Wilks (1938) proves that $\Delta\chi^2$ for an $N_p$-dimensional
parameter space will be distributed according to a $\chi^2$-distribution
with $N_p$ degrees of freedom.  One can perform the same integral discussed
above and set $\Delta\chi^2$ to that value which encloses the desired confidence
limit.  This is the well-known Likelihood Ratio Test \cite{np}.
In this case, the user is relying on APS to eventually find the true
$\chi^2_\text{min}$ of the likelihood surface being explored.  
Section \ref{sec:simplex} below outlines an additional means of search
by which our APS is capable of efficiently locating $\chi^2_\text{lim}$.

\subsection{Modified search algorithms}

Bryan {\it et al.} (2007) show that the APS search algorithm as presented in steps (1A-3A)
is robust against multi-modal
likelihood functions, using it to identify a disjoint region of high likelihood in
cosmological parameter space as constrained by the 1-year WMAP CMB anisotropy power
spectrum.  This is the principal advantage to selecting points for evaluation according
to the statistic (\ref{eqn:sstat}).  Unfortunately, it is possible that the focus on 
exploring
unknown regions of parameter space might slow down convergence in the case of uni-modal
likelihood functions.  Therefore, we propose the following modified search
to ensure that it adequately characterizes known regions of high likelihood while
simultaneously searching for new regions to explore.

These modified searches will require APS to keep track of information it learns about the
$\vec{\theta}\rightarrow\chi^2$ fuction as it searches.  
To facilitate this discussion, we introduce the folling notations.
Symbols denoted with $\vec{x}$ represent points in parameter space.  Variables denoted with
$\{\hat{X}\}$ denote sets of points in parameter space.  Variables denoted $\{X\}$ are sets
of numbers stored by APS.

We are greatly indebted to Eric Linder of the Lawrence Berkeley National
Laboratory who first suggested to us the modifications described below
in sections \ref{sec:focus} and \ref{sec:simplex}.

\subsubsection{Characterization of known low-$\chi^2$ regions}
\label{sec:focus}

As it searches, APS will keep track of a list of all of the discovered points
which are believed to be local minima of $\chi^2$ 
(how it finds these will be described below
in Section \ref{sec:simplex}).  These points will be stored in the set
$\{\hat{\Gamma}\}$.  After each cycle of the main algorithm (steps 1A-3A above),
our version of APS attempts ``focused searches'' about the points in $\{\hat{\Gamma}\}$.  

For each point $\vec{M}$ in $\{\hat{\Gamma}\}$, APS performs the following:
\begin{itemize}
\item (1B) If fewer than $N_p$ focused searches have been proposed around this minimum, 
randomly choose a point from a small sphere surrounding $\vec{M}$ in parameter space and 
evaluate $\chi^2$ at that point.  
\\
\item(2B) If the point does not satisfy $\chi^2=\chi^2_\text{lim}$ (which is highly probable), 
use bisection
along the line connecting the proposed point to $\vec{M}$ to find a point which does
satisfy $\chi^2=\chi^2_\text{lim}$, i.e. find a point on that line with
$\chi^2<\chi^2_\text{lim}$ -- often $\vec{M}$ itself -- and a point along that line
with $\chi^2>\chi^2_\text{lim}$ and iteratively step half the distance between them until
arriving at the desired $\chi^2_\text{lim}$ \cite{minuit}.  The points discovered this way \
with $\chi^2=\chi^2_\text{lim}$ are stored in the set $\{\hat{B}(\vec{M})\}$ of 
``boundary points'' about $\vec{M}$ (each $\vec{M}$ in $\{\hat{\Gamma}\}$ will have its own
set $\{\hat{B}(\vec{M})\}$).  
\\
\item(3B) If $N_p$ boundary points have been found for $\vec{M}$, skip steps (1B) and (2B).
Instead, examine the points in $\{\hat{B}(\vec{M})\}$ and identify the points
corresponding to both the maximum and minimum values of each parameter $\theta_i$.  These
points will be referred to as $\{\hat{B}_\text{MaxMin}(\vec{M})\}$.  
There will be $2\times N_p$ points in $\{\hat{B}_\text{MaxMin}(\vec{M})\}$, one for
the maximum and one for the minimum in each of our $N_p$ parameters.  
\\
\item(4B) For each point $\vec{b}$ in $\{\hat{B}_\text{MaxMin}(\vec{M})\}$, 
propose some number $N_v$ of new points (we have chosen
$N_v=20$ for the sake of speed) $\vec{v} = \vec{b}+\vec{\epsilon}$ where $\vec{\epsilon}$ is a
random, small vector in parameter space constrained to be perpendicular to
$\vec{b}-\vec{M}$.  There will now be $2\times N_p \times N_v$ points $\vec{v}$
($N_v$ such points for each of the $2\times N_p$ extremal boundary points in
$\{\hat{B}_\text{MaxMin}(\vec{M})\}$).
\\
\item(5B) At each point $\vec{v}$ above, evaluate equation (\ref{eqn:sstat}).  
Evaluate $\chi^2$ at the point $\vec{v}$ which maximizes $S$. If
$\chi^2 \neq \chi^2_\text{lim}$, bisection is again used and a new boundary point is found.
\end{itemize}

To reiterate, our version of APS alternates between performing steps (1A-3A) and steps
(1B-5B).  If there are $N_m$ distinct known local minima in $\{\hat{\Gamma}\}$, 
our algoritm will perform
$N_m$ iterations of steps (1A-3A), and then perform steps (1B-5B) once for each
$\vec{M}$ in $\{\hat{\Gamma}\}$.

\subsubsection{Extension of the $\chi^2=\chi^2_\text{lim}$ contour}
\label{sec:unitSphere}

The modification suggested in section \ref{sec:focus} is designed to explore 
$\chi^2=\chi^2_\text{lim}$ surface by starting from its extreme corners and stepping
perpendicularly to its ``radii'' (the vector $\vec{b}-\vec{M}$ would be a radius of
$\chi^2=\chi^2_\text{lim}$ described a sphere).  It is, however, also possible
that the $S$-maximization in steps (1A-3A) could also learn something about the
shape of the $\chi^2=\chi^2_\text{lim}$ contour, even if it does not place
points directly on that contour.  To explore that possibility, we add the
following steps to the end of the sequence (1A-3A).

\begin{itemize}
\item(1C) For each point $\vec{M}$ in $\{\hat{\Gamma}\}$, store, 
in addition to the boundary
points $\{\hat{B}(\vec{M})\}$, the projection of those boundary points
onto a unit sphere in parameter space, i.e. for each point $\vec{b}$ in $\{\hat{B}(\vec{M})\}$, draw a line
connecting $\vec{b}$ with $\vec{M}$ and keep track of
where a unit sphere centered on $\vec{M}$ intersects that line.  Call this set
of projectsion $\{\hat{U}\}$.
\\
\item(2C) Call the point for which $\chi^2$ has been evaluated in step (3A)
$\vec{P}$., find the
point $\vec{M}$ in $\{\hat{\Gamma}\}$ that is nearest to $\vec{P}$ in parameter space.
\\
\item(3C) Find the
projection of $\vec{P}$ onto the unit sphere surrounding $\vec{M}$.  
Call this projected point $\vec{p}$.  Find the point $\vec{u}$ in
$\{\hat{U}\}$ that is nearest to $\vec{p}$.  Record the parameter space 
distance between $\vec{u}$ and $\vec{p}$.
Add this distance to the set $\{D\}$ ($\{D\}$ will be a running list
of the $\vec{u}-\vec{p}$ distance for all points $\vec{P}$ found by steps
1A-3A, regardless of which point in $\vec{M}$ they are nearest).
\\
\item(4C) If the $\vec{p}-\vec{u}$ distance is greater than the 2/3 quantile of $\{D\}$,
then perform bisection between $\vec{M}$ and the point evaluated in step (3A).
\end{itemize}

This way, points on the $\chi^2=\chi^2_\text{lim}$ contour can be found whose
directions away from $\vec{M}$ in parameter space have not yet been explored.
Because we expect the unit spheres surrounding the points $\vec{M}$ to become more densely
populated as the algorithm runs, we periodically delete the list $\{D\}$ so that
new directions away from $\vec{M}$ can still be explored.

\subsubsection{Function minimization}
\label{sec:simplex}

The final modification to the original APS algorithm which we propose in this
work is to utilize function minimization to identify local minima in $\chi^2$.
One of the principal drawbacks of the algorithm presented in steps (1A-3A) is
that it makes no account of ``near misses.''  If step (3A) finds that $\chi^2$
is lower than average but still greater than $\chi^2_\text{lim}$, the original
APS algorithm does nothing with that information.  We have tried to improve upon
that with our incorporation of bisection in sections \ref{sec:focus} and
\ref{sec:unitSphere} above.  Our final innovation is to allow APS to use points
with 
\begin{eqnarray}
\chi^2_\text{lim}&\le&\chi^2\le\chi^2_\text{threshold}\label{eqn:simplexCriterion}\\
\chi_\text{threshold}&=&\chi^2_\text{min}+G\times(\chi^2_{1/10}-\chi^2_\text{min})
\label{eqn:threshold}
\end{eqnarray}
to seed a simplex function minimizer as developed by Nelder and Mead (1965).
Here, $G$ is a user-set parameter, $\chi^2_\text{min}$ is the minimum discovered
$\chi^2$ value and $\chi^2_{1/10}$ is the 1/10 quantile of the $\chi^2$ values
discovered by steps (1A-3A) (not those discovered by any of the modifications
discussed in sections \ref{sec:focus}, \ref{sec:unitSphere} or
\ref{sec:simplex}).

Because the Nelder-Mead simplex minimizer can be very time intensive, we balance our
algorithm by demanding that the number of $\chi^2$ evaluations devoted to all
other variants of search (i.e. steps (1A-3A) and (1B-5B)) equal the number of
$\chi^2$ evaluations devoted to the simplex minimizer before a new simplex
minimization can occur.

If the simplex minimizer converges to a point such that
$\chi^2<\chi^2_\text{lim}$, our APS determines whether or not it
has found a new local minimum to be added to $\{\hat{\Gamma}\}$ by comparing the end
point $\vec{y}$ of the minimization to each of the points already in
$\{\hat{\Gamma}\}$.  For each $\vec{M}$ in $\{\hat{\Gamma}\}$, APS calculates the mid-point
\begin{equation}
\label{eqn:midpt}
\vec{m}=0.5\times\left(\vec{y}+\vec{M}\right)
\end{equation}
and $\vec{M}$.  If, for all $\vec{M}$, 
$\chi^2$ at $\vec{m}$ is greater
than $\chi^2_\text{lim}$, then it is deemed that $\vec{\gamma}$ is a new local
minimum and is added to $\{\hat{\Gamma}\}$.  If not, $\vec{\gamma}$ is assumed to be a
part of the one of the previously identified loci of low $\chi^2$ and it is not
added to $\{\hat{\Gamma}\}$.  Regardless of whether or not $\vec{y}$ passes the test in
equation (\ref{eqn:midpt}), $\vec{y}$ is added to the set $\{\hat{Y}\}$ of all
simplex end points discovered by APS.

As a further guard against excessive simplex searching, all points $\vec{w}$ 
that pass the test in equation(\ref{eqn:simplexCriterion}) are further required
to satisfy the condition
\begin{eqnarray}
\chi^2(\vec{z})&\le&\chi^2(\vec{w})-0.25\times\left(\chi^2(\vec{w})-\chi^2(\vec{y})\right)\\
\vec{z}&\equiv&0.5\times\left(\vec{w}+\vec{y}\right)\\
\vec{y}&\in&\{\hat{Y}\}
\end{eqnarray}
before it is allowed to seed a new simplex search.  In this way, our APS avoids seeding simplex
searches with points that are inside of already known regions of low $\chi^2$.

\subsection{Code implementing APS}
\label{sec:user}

We make code implementing our extended APS for 
an arbitrary $\vec{\theta}\rightarrow\chi^2$
function available at \verb|https://github.com/uwssg/APS/|.
The code is written in C++.  It provides the option for several
user-specified parameters.  We list them below along with the C++ 
methods (belonging to the class \verb|aps|) used to set them.  
This code includes several user-specified parameters which we
outline below.

\begin{itemize}
\item The parameter $G$ from equation (\ref{eqn:threshold}) above can be set using the method
\verb|aps.set_grat|
\\
\item $\chi^2_\text{lim}$ can be manually set using the method \verb|aps.set_target|
\\
\item $\Delta \chi^2$ can be set in the constructor for the class \verb|aps|.
\\
\item The number $N_g$ of nearest neighbors used in step 2A can be set in the constructor
for the class \verb|aps|
\\
\item $N_c$ from step 2A can be set using the method \verb|aps.set_samples|
\\
\item The minimum and maximum bounds in parameter space can be set using the 
method \verb|aps.set_max| and \verb|aps.set_min|.  These bounds will constrain the range of
points sampled in step 2A.  It will not contraint searches performed with bisection or the 
simplex minimizer.
\\
\item If there is a characteristic length scale associated with a parameter, this can be set
using the method \verb|aps.set_characteristic_length|.  This length scale will be used to
normalize all distances in parameter space.  If it is not set, it will default to the
difference between the maximum and minimum values set by \verb|aps.set_max| and
\verb|aps.set_min|.
\end{itemize}

More complete documentation is made available with the source code.

\subsection{Interpreting APS outputs}
\label{sec:bayes}

Originally, APS was conceived as a way to draw Frequentist confidence limits. 
$\chi^2_\text{lim}$ is determined, either in a vacuum or as some $\Delta\chi^2$ away from
$\chi^2_\text{min}$ (see Appendix \ref{sec:frequentism} for a more thorough discussion).
This is by far the most direct way to interpret APS outputs: a list of points is given with
corresponding values of $\chi^2$.  All of the points with $\chi^2\le\chi^2_\text{lim}$ are
taken to circumscribe the desired confidence limit.  There is, however, also a way to use the
outputs from APS to draw a Bayesian credible limit.

As discussed in the introduction, Bayesian parameter constraints assume that there is some
probability distribution (``the posterior'') over $\vec{\theta}$.  
$(1-\alpha)\%$ credible limits are drawn by
integrating over this probability distribution and selecting that region which contains
$(1-\alpha)\%$ of the total probability.  This integral can be very time-consuming if the
dimensionality of $\vec{\theta}$ is large.  Markov Chain Monte Carlo methods overcome this
difficulty by randomly drawing samples from the posterior and assuming that, if enough samples
are drawn, the distribution of those samples will be sufficient to reconstruct the attributes
of the full posterior probability density.

APS outputs cannot be used this way, since APS does not select points to evaluate
probabilisticaly, but in an attempt to recreate the behavior of the $\chi^2$ function in
interesting regions of parameter space.  We can still, however, use these points to approximate
the integral over the posterior probability density.  The algorithm we propose is as follows.
The total set of points sampled by APS will be referred to below as $\{\hat{A}\}$.

\begin{itemize}
\item(1D) For each point $\vec{a}$ in $\{\hat{A}\}$ initialize two vectors $\vec{a}_\text{max}$ and
$\vec{a}_\text{min}$.

\item(2D) Select the $3N_p+1$ nearest neighbor points to $\vec{a}$ from
$\{\hat{A}\}$ (this will obviously include $\vec{a}$ itself, which can be discarded).  Call
this set of nearest neighbors $\{\hat{N}\}$.
\\
\item(3D) For each point $\vec{n}$ in $\{\hat{N}\}$, calculate the vector
$\vec{\delta}=\|\vec{n}-\vec{a}|$ (where the vertical bars denote the absolute value of the
components of $\vec{\delta}$, not any kind of vector normalization).  For each component
$\delta_i$ of $\vec{\delta}$, starting with the largest and working towards the smallest:
\\
\begin{itemize}
\item(3Da) If $n_i>a_i$ and $a_{\text{max},i}$ has not been set, set $a_{\text{max},i}=n_i$.
\\
\item(3Db) If $n_i<a_i$ and $a_{\text{min},i}$ has not been set, set $a_{\text{min},i}=n_i$.
\\
\item(3Dc) If both $a_\text{max,i}$ and $a_\text{min,i}$ have already been set, move on to the
next largest component of $\vec{\delta}$ and repeat at step (3Da).
\end{itemize}
\item(4D) Choose the next neighbor $\vec{n}$ and repeat from step (3D). Continue until all
$a_{\text{max},i}$ and $a_{\text{min},i}$ have been set.  Once all $a_{\text{max},i}$ and
$a_{\text{min},i}$ have been set, move on to step (5D).
\\
\item(5D) $\vec{a}_\text{max}$ an $\vec{a}_\text{min}$ should now describe an asymmetric,
rectangular hyperbox about $\vec{a}$.  To symmetrize it, step through all $N_p$ dimensions
and reset either $a_{\text{max},i}$ or $a_{\text{min},i}$ so that 
$d_\text{min}=a_i-a_{\text{min},i}$ and
$d_\text{max}=a_{\text{max},i}-a_i$ are equal to the minimum of $\{d_\text{max},d_\text{min}\}$
from the asymmetric hyperbox.
\end{itemize}

At this point, each $\vec{a}$ in $\{\hat{A}\}$ should have a corresponding symmetric hyperbox
surrounding it.  To integrate the posterior probability density, we will assume that all of the
parameter space points in each hyperbox correspond to a $\chi^2$ equal to $\chi^2(\vec{a})$.
The posterior probability associated with each box is then
\begin{eqnarray}
P(\vec{a})&=&\frac{V(\vec{a})}{P_\text{tot}}\times\exp\left[-\chi^2(\vec{a})/2\right]
\label{eqn:posterior}\\
P_\text{tot}&\equiv&\sum_{\vec{a}} V(\vec{a})\times\exp\left[-\chi^2(\vec{a})/2\right]
\end{eqnarray}
where $V(\vec{a})$ is the hypervolume associated with the hyperbox surrouning $\vec{a}$.
To plot the $(1-\alpha)\%$ credible limit, one need only plot the points in hyperspace that
enclose the lowest-$\chi^2$ hyperboxes containing $(1-\alpha)\%$ of the total probability.
We test this algorithm on a function with an exactly known credible limit
in Section \ref{sec:toy} and on a real, physical dataset in Section \ref{sec:wmap}.  In 
Section \ref{sec:toy}, we find that our results agree with the known credible limit.
In Section \ref{sec:wmap}, we find that our results agree with results derived from traditional
Markov Chain Monte Carlo methods.

The functionality described by this section is provided in by the classes defined in the source
code file \verb|aps_extractor.cpp| in our software package.

\section{Gaussian Processes}
\label{sec:gp}

Step (2A) of \APS requires that a prediction $\mu$ be made regarding the value
of $\chi^2$ at points as yet unsampled by the algorithm.  An uncertainty $\sigma$
must also be assigned to this prediction.  APS writ large is agnostic regarding
the method used to find $\mu$ and $\sigma$.  We choose to use a Gaussian process.

Gaussian processes take noisy, sparse, measurements of an unknown function (in our
case, $\vec{\theta}\rightarrow\chi^2$ at
the points in parameter space already sampled by APS) and use them to
make inferences about unmeasured values of the function by assuming that the
function is a random process.  Gaussian processes have already found use in physics
and astronomy constructing non-parametric models of the cosmic expansion
\cite{ericgp}, interpolating the point spread function across 
astronomical images
\cite{psf}, and interpolating models of the non-linear matter
power spectrum from N-body simulations \cite{habib}.
They are useful in the current context because, not only do they
give robust predictions $\mu$ for the unmeasured values of the
function being modeled, but they also
provide well-motivated prediction uncertainties $\sigma$, a requirement of
equation (\ref{eqn:sstat}).  We present below an introduction to the
formalism of Gaussian processes drawn heavily
from Chapter 2 and Appendix A 
of \cite{gp}.

Gaussian processes use the sampled data $\{\vec{\theta}^{(i)},\chi^{2,(i)}\}$, 
where $i$ indexes over all of the sampled points, to predict
$\mu=\chi^{2,(q)}$ at the candidate point 
$\vec{\theta}^{(q)}$ by assuming that the function
$\vec{\theta}\rightarrow\chi^2$ represents a sample drawn from a random process 
distributed
across parameter space.  At each point in parameter space, 
$\chi^2$ is assumed to be
distributed according to a normal distribution with mean 
$\bar{\chi}^2$
and variance dictated by the ``covariogram''
$C_{ij}(\vec{\theta}^{(i)},\vec{\theta}^{(j)})$.
If we have $N_g$ measurements of $\chi^2$, then we assume
they are distributed according to
\begin{equation}
\label{eqn:likelihood}
P(\vec{\chi}^2)=
\frac{\exp\bigg[-\frac{1}{2}K^{-1}\sum_{i,j}^{N_g}
(\chi^{2,(i)}-\bar{\chi}^2)C^{-1}_{ij}
(\chi^{2,(j)}-\bar{\chi}^2)\bigg]}{\sqrt{2\pi}^{N_g}\text{det}|KC|^{N_g/2}}
\end{equation}
$C_{ij}$ is a function assumed by the user encoding how variations in
$\chi^2$ at one point in parameter space affect variations in 
$\chi^2$ at other
points in parameter space.  $K$ is a parameter controlling the
normalization of $C_{ij}$.
The diagonal elements $C_{ii}=1+\sigma^2_{ii}$ where 
$\sigma^2_{ii}$ is the intrinsic variance in the value
of $\chi^2$ at a single point $\vec{\theta}^{(i)}$.  
Rasmussen and Williams (2006) treat 
the special case
$\bar{\chi}^2=0$ and find (see their equations 2.19, 2.25 and 2.26)
\begin{eqnarray}
\mu&=&\sum_{i,j} C_{qi}C^{-1}_{ij}\chi^{2,(j)}\nonumber\\
\sigma^2&=&C_{qq}-\sum_{i,j}C_{qi}C^{-1}_{ij}C_{jq}\label{sig}
\end{eqnarray}
where the sums over $i$ and $j$ are sums over the sampled points 
in parameter space.  $C_{iq}$ relates the sampled point $i$ to the candidate
point $q$.
We do not wish to assume that the mean value of $\chi^2$ is zero everywhere.
Therefore, we modify the equation for $\mu$ to give
\begin{equation}
\label{mu}
\mu=\bar{\chi}^2+\sum_{i,j} C_{qi}C^{-1}_{ij}
(\chi^{2,(j)}-\bar{\chi}^2)
\end{equation}
where $\bar{\chi}^2$ is the algebraic mean of the 
sampled $\chi^{2,(i)}$.
Note the similarity to a multi-dimensional Taylor series expansion with the
covariance matrix playing the role of the derivatives.
Equation (\ref{sig}) differs from equation (6) in \cite{bryan} because
they used the semivariance
$$\gamma_{ij}=\text{var}[\chi^2(\vec{\theta}^{(i)})
-\chi^2(\vec{\theta}^{(j)})]$$
in place of the covariance $C_{ij}$.  In practice, the two assumptions result in
equivalently valid $\mu$ and $\sigma$.

The form of the covariance function 
$C_{ij}(\vec{\theta}^{(i)},\vec{\theta}^{(j)})$ must be
assumed.  One possibility, 
taken from equation 4.9 of Rasmussen and Williams (2006) is
\begin{equation}
\label{eqn:GaussianCovar}
C_{ij}=\exp\left[-\frac{1}{2}D^2_{ij}\right]
\end{equation}
where $D_{ij}$ is the normalized distance in parameter space between the points 
$\vec{\theta}^{(i)}$
and $\vec{\theta}^{(j)}$.
\begin{equation}
D^2_{ij}\equiv\sum_n^{N_p}
\left(\frac{\theta^{(i)}_n-\theta^{(j)}_n}{\ell_n\times(\text{max}_n-\text{min}_n)}\right)^2
\end{equation}
where $\text{max}_n-\text{min}_n$ is the difference between the maximum and minimum values of
the $n$th parameter taken from the $N_g$ points used to seed the Gaussian process and $\ell_n$
is a ``hyper-parameter'' associated with the $n$th parameter.  We both the normalization
paramter $K$ below in section \ref{sec:kriging}.  We discuss the setting of
hyper-parameters in section \ref{sec:hyperparams}.
The exponential form of $C_{ij}$ quantifies the assumption 
that distant points should
not be very correlated.  We will use this covariogram in
our tests in Section \ref{sec:toy} and \ref{sec:wmap}.

Another possibility (equation 4.29 of Rasmussen and Williams) is
\begin{eqnarray}
C_{ij}=\phantom{\sin
\bigg[\frac{2(\sigma_0+\tilde{\vec{\theta}}^{(i)}\cdot\tilde{\vec{\theta}}^{(j)})}
{\sqrt{(1+2(\sigma_1+\tilde{\vec{\theta}}^{(i)}\cdot\tilde{\vec{\theta}}^{(i)}))
(1+2(\sigma_1+\tilde{\vec{\theta}}^{(j)}\cdot\tilde{\vec{\theta}}^{(j)}))}}\bigg]}
\label{eqn:NNCovar}\\
\frac{2}{\pi}\sin^{-1}
\bigg[\frac{2(\sigma_0+\tilde{\vec{\theta}}^{(i)}\cdot\tilde{\vec{\theta}}^{(j)})}
{\sqrt{(1+2(\sigma_1+\tilde{\vec{\theta}}^{(i)}\cdot\tilde{\vec{\theta}}^{(i)}))
(1+2(\sigma_1+\tilde{\vec{\theta}}^{(j)}\cdot\tilde{\vec{\theta}}^{(j)}))}}\bigg]
\nonumber
\end{eqnarray}
corresponding to the covariance function of a neural network with a single
hidden layer.  In this function $\tilde{\vec{\theta}}$ is the vector of
parameters $\vec{\theta}$ recentered and renormalized relative to the span
$(\text{max}_n-\text{min}_n)$ in each dimension.  $\sigma_0$ and $\sigma_1$
are hyperparameters.  We also consider this
covariogram in Section \ref{sec:wmap} and find no appreciable difference
in performance when compared to the use of covariogram (\ref{eqn:GaussianCovar}).

\subsection{Normalizing the covariogram}
\label{sec:kriging}

The normalization constant $K$ in equation (\ref{eqn:likelihood}) --known as the
``Kriging parameter'' for the geophysicist who pioneered the overall method -- 
also
must be assumed.  Note that $K$ also multiplies the diagonal elements
$C_{ii}$.  Determining the value of $K$ 
is somewhat problematic because, examining equation
(\ref{mu}), one sees that the factors of $K$ and $K^{-1}$
cancel out of
the prediction $\mu$, so that the assumed value of $K$ has no effect on the accuracy
of the prediction.  
If the opposite had been true, one could heuristically set $K$ to
maximize the accuracy of $\mu$.  
Instead, we set $K$ to the value that maximizes the likelihood
of the input data, i.e. the $N_g$ nearest neighbors in parameter space
used to perform the Gaussian Process inference.

Recall from equation (\ref{eqn:likelihood}) that we are treating our
input data as though they were samples drawn from a probability distribution.
To maximize the probability of the observed data (i.e. maximizing the value
of $P(\vec{\chi}^2)$) we must set $K$ equal to
\begin{equation}
\label{eqn:kriging}
K=\frac{\sum_{i,j}^{N_g}(\chi^{2,(i)}-\bar{\chi}^2)
C^{-1}_{ij}(\chi^{2,(j)}-\bar{\chi}^2)}
{N_g}
\end{equation}
We adopt this assumption throughout this work.

Figure \ref{fig:gp} applies the Gaussian process of equations (\ref{sig}), 
(\ref{mu}), and (\ref{eqn:kriging}) with covariogram (\ref{eqn:GaussianCovar})
to a toy one-dimensional function.  Inspection shows many desirable
behaviors in $\mu$ and $\sigma$.  
As $\theta^{(q)}$ approaches the sampled points $\theta^{(i)}$, $\mu$
approaches $\chi^{2,(i)}$ and $\sigma$ approaches zero.  
Closer to a sampled point,
the Gaussian process knows more about the true behavior of the function. 
Far from the $\theta^{(i)}$, $\sigma$ is larger, and the $S$ statistic in
equation (\ref{eqn:sstat}) will induce the \APS
algorithm to examine the true value of $\chi^2$.

\begin{figure}
\includegraphics[scale=0.3]{gp_example_1307.eps}
\caption{
A one-dimensional example of prediction using Gaussian processes.  
The black curve is the function
being considered.  The crosses are the points at which it has been sampled.  The
green curve is the resulting prediction and the red curves represent the 
1- and
2-$\sigma$ uncertainty bounds.  We set $K$ according to 
equation (\ref{eqn:kriging}) wth $N_g=5$ for the 5 sampled points.
}
\label{fig:gp}
\end{figure}

\subsection{Setting hyper-parameters}
\label{sec:hyperparams}

The covariance functions $C_{ij}$ presented above (indeed, any covariance function)
each contain arbitrary hyper-parameters whose values can be set to fine-tune
the Gaussian Process model ($\ell_n$ in equation \ref{eqn:GaussianCovar}
and $\{\sigma_0,\sigma_1\}$ in equation \ref{eqn:NNCovar}).  
In order to set the hyper-parameters used in the APS Gaussian Process, we randomly
select a set $\{\hat{V}\}$ of up to 3000 points from those already sampled by APS 
and use these to construct
the error function $E(\vec{h})$ where $\vec{h}$ is a vector specifying the hyper-parameters
required by the chosen $C_{ij}$.  $E$ is defined as
\begin{equation}
\label{eqn:hyperparamError}
E(\vec{h})=\sum_{\vec{v}\in\{\hat{V}\}}\left(\mu(\vec{v},\vec{h})-\chi^2(\vec{v})\right)^2
\end{equation}
where $\mu(\vec{v},\vec{h})$ is calculated according to equation \ref{mu} using
hyper-parameters $\vec{h}$ (and ignoring $\vec{v}$ as its own nearest neighbor) and
$\chi^2(\vec{v})$ is already known, since $\vec{v}$ has been chosen from the points previously
sampled by APS.  If $\vec{h}$ has more than two dimensions, we minimize $E$ on $\vec{h}$-space
using the simplex
minimizer of Nelder and Mead (1965).  Otherwise, we do a simple grid search for the value of
$\vec{h}$ which minimizes $E$.  This optimal value of $\vec{h}$ is used to set
the hyper-parameters for our Gaussian Process model.

As APS samples more points in parameter space, we expect it to learn more about the
$\vec{\theta}\rightarrow\chi^2$ function and thus be capable of a more accurate Gaussian
Process, so we allow it to periodically reset the hyper-parameters $\vec{h}$.  However,
minimizing $E$ can be a very time-consuming process, so we have APS keep track of the amount of
time elapsed since the algorithm began.  We divide this value by the number of calls made to
the $\vec{\theta}\rightarrow\chi^2$ function and compare this amortized time to the average
amount of time spent on one call to $\vec{theta}\rightarrow\chi^2$.  APS us only allowed to
reset its hyper-parameters if the difference between these two times is less than 0.1 second.

\section{A Toy Example}
\label{sec:toy}

We will now test the ability of APS to learn the confidence limits of an
artificial $\chi^2$ function and compare it to that of MCMC driven by the
traditional Metropolis-Hastings algorithm \cite{mcmc,cosmomc}.

The algorithm (1A)-(3A) in Section \ref{sec:algorithm} 
was originally presented and tested against the 1-year data
release from the WMAP satellite in Bryan {\it et al}. (2007).  They found that
the algorithm detected a second locus of low $\chi^2$
that had gone undetected by previous, MCMC-based analyses of the data.  
This second fit to the data has disappeared as the signal-to-noise
ratio of the data has improved (as we will see in Section \ref{sec:wmap}),
therefore, we will use for our test a toy $\chi^2$ function
with multiple locii of low $\chi^2$.  

We construct our toy $\chi^2$ as a series of low-$\chi^2$ ellipses in a 5-dimensional
parameter space (we choose a 5-dimensional parameter space because it allows us to
reasonably show all of the 2-dimensional sub-spaces in the plots which follow).
We consider functions with 2, 3, and 4 discrete minima in $\chi^2$.
The location of the low-$\chi^2$ regions are generated using a pseudo random number
generator taken from Marsaglia (2003) with parameters taken from Press {\it et al.} (2007).
$\chi^2(\vec{\theta})$ for this toy function is calculated as 
\begin{equation}
\label{eqn:toychi}
\chi^2(\vec{\theta})=
\sum_i\left(\frac{\theta_i-c_{\text{nearest},i}}{w_{\text{nearest},i}}\right)^2
\end{equation}
where $c_{\text{nearest},i}$ is the $i$th coordinate of the center of the nearest low-$\chi^2$
region and $w_{\text{nearest},i}$ is the (pseudo-randomly generated) width of that ellipse in
the $i$th coordinate.  We refrain from introducing any interesting parameter degeneracies so
that the $\chi^2(\vec{\theta})$ function remains integrable in 2-dimensional sub-spaces.
This gives us a hard-truth ``control'' against which to evaluate our test APS runs below.
Readers wishing to re-create this $\chi^2$ function can find it in our software
package as the class \verb|ellipses_integrable| in the source code files \verb|chisq.h| and
\verb|chisq.cpp|.

To test the performance of our APS algorithm, we ran APS on iterations of this
$\vec{\theta}\rightarrow\chi^2$ function with 2, 3, and 4 discrete minima in $\chi^2$.  We
allowed APS to sample a total of 10,000 points in the 5-dimensional parameter space.  
We had APS adaptively set $\chi^2_\text{lim}=\chi^2_\text{min}+11.0$ -- $11.0$ being the $95\%$
confidence limit bound for $\chi^2$ probability distribution with 5 degrees of freedom.
Figures
\ref{fig:toyFreq2}, \ref{fig:toyFreq3}, and \ref{fig:toyFreq4} plot the
$\chi^2\le\chi^2_\text{lim}$ points found by APS (black points) against the
known $\Delta\chi^2=11.0$ contours of the function (red contours).  Obviously, APS did a very
good job of exploring all of the available low-$\chi^2$ regions.

As mentioned above, because our toy $\chi^2$ function is constructed without any
parameter degeneracies, we are able to directly integrate the posterior and
determine the true Bayesian credible limits in our parameter space.  Figures
\ref{fig:toyBayes2}, \ref{fig:toyBayes3}, and \ref{fig:toyBayes4} plot the true
95\% credible limits in all 10 2-dimensional sub-spaces of our 5-dimensional
parameter space (red contours).  They also show the APS-determined credible
limits as described in Section \ref{sec:bayes} (black points).  The
APS-determined credible limits correspond well with the true credible limits,
giving us confidence in APS's ability to map out Bayesian credible limits as well
as Frequentist confidence limits.

The tests above demonstrate APS's ability to correctly determin credible and
confidence limits.  They do not, however, address the question of APS's chief
advantage over Markov Chain Monte Carlo methods: its ability to successfully
locate all of the low-$\chi^2$ regions in a given parameter space.  To
demonstrate this, we ran APS 100 times on each of the likelihood functions
plotted in Figures \ref{fig:toyFreq2}-\ref{fig:toyFreq4}.  Each time, the
pseudo-random number generator driving APS was given a different seed.  In all
cases, APS successfully located all of the local minima in $\chi^2$, usually in
fewer than 10,000 evaluations of $\chi^2$ (one case of the 2-mode $\chi^2$
function required 10,352 evaluations to find both low-$\chi^2$ regions; one
required 12,549 evaluations).  For comparison, we also ran 100 Markov Chain
Monte Carlo searches on these $\chi^2$ function, each run consisting of five
independent chains.  These searches were not nearly as successful as APS at
locating all of the low-$\chi^2$ regions.  In the case of the 2-mode function,
50 of the 100 MCMC searches failed to identify both modes.  In the case of the
3-mode function, 56 of the 100 MCMC searches failed to identify all of the
modes.  In the case of the 4-mode function, 87 of the 100 MCM searches failed to
identify all of the modes.  This is not to say that APS is perfect at
identifying all of the modes of a likelihood function.  If we reduce the
characteristic widths $w_{j,i}$ in equation (\ref{eqn:toychi}), we enter a regime
in which neither APS nor MCMC is capable of reliably,
identifying all of the modes.  However, this test does indicate that APS is more
robust than MCMC against multi-modal likelihood functions, as was first
demonstrated by Bryan {\it et al}. (2007).

\begin{figure*}
\subfigure[]{
\includegraphics[scale=0.15]{ellipse_d5_c2_0_1_frequentist.eps}
}
\subfigure[]{
\includegraphics[scale=0.15]{ellipse_d5_c2_0_2_frequentist.eps}
}
\subfigure[]{
\includegraphics[scale=0.15]{ellipse_d5_c2_0_3_frequentist.eps}
}
\subfigure[]{
\includegraphics[scale=0.15]{ellipse_d5_c2_0_4_frequentist.eps}
}
\subfigure[]{
\includegraphics[scale=0.15]{ellipse_d5_c2_1_2_frequentist.eps}
}
\subfigure[]{
\includegraphics[scale=0.15]{ellipse_d5_c2_1_3_frequentist.eps}
}
\subfigure[]{
\includegraphics[scale=0.15]{ellipse_d5_c2_1_4_frequentist.eps}
}
\subfigure[]{
\includegraphics[scale=0.15]{ellipse_d5_c2_2_3_frequentist.eps}
}
\subfigure[]{
\includegraphics[scale=0.15]{ellipse_d5_c2_2_4_frequentist.eps}
}
\subfigure[]{
\includegraphics[scale=0.15]{ellipse_d5_c2_3_4_frequentist.eps}
}
\caption{
This plot shows all 10 2-dimensional sub-spaces of our 5-dimensional toy likelihood
function with 2 minima in $\chi^2$.  The black points are $\chi^2\le\chi^2_\text{lim}$
found by APS after 10,000 samplings.  The red contours are the known $\chi^2=\chi^2_\text{lim}$
contours of the function.
}
\label{fig:toyFreq2}
\end{figure*}

\begin{figure*}
\subfigure[]{
\includegraphics[scale=0.15]{ellipse_d5_c3_0_1_frequentist.eps}
}
\subfigure[]{
\includegraphics[scale=0.15]{ellipse_d5_c3_0_2_frequentist.eps}
}
\subfigure[]{
\includegraphics[scale=0.15]{ellipse_d5_c3_0_3_frequentist.eps}
}
\subfigure[]{
\includegraphics[scale=0.15]{ellipse_d5_c3_0_4_frequentist.eps}
}
\subfigure[]{
\includegraphics[scale=0.15]{ellipse_d5_c3_1_2_frequentist.eps}
}
\subfigure[]{
\includegraphics[scale=0.15]{ellipse_d5_c3_1_3_frequentist.eps}
}
\subfigure[]{
\includegraphics[scale=0.15]{ellipse_d5_c3_1_4_frequentist.eps}
}
\subfigure[]{
\includegraphics[scale=0.15]{ellipse_d5_c3_2_3_frequentist.eps}
}
\subfigure[]{
\includegraphics[scale=0.15]{ellipse_d5_c3_2_4_frequentist.eps}
}
\subfigure[]{
\includegraphics[scale=0.15]{ellipse_d5_c3_3_4_frequentist.eps}
}
\caption{
This plot shows all 10 2-dimensional sub-spaces of our 5-dimensional toy likelihood
function with 3 minima in $\chi^2$.  The black points are $\chi^2\le\chi^2_\text{lim}$
found by APS after 10,000 samplings.  The red contours are the known $\chi^2=\chi^2_\text{lim}$
contours of the function.
}
\label{fig:toyFreq3}
\end{figure*}

\begin{figure*}
\subfigure[]{
\includegraphics[scale=0.15]{ellipse_d5_c4_0_1_frequentist.eps}
}
\subfigure[]{
\includegraphics[scale=0.15]{ellipse_d5_c4_0_2_frequentist.eps}
}
\subfigure[]{
\includegraphics[scale=0.15]{ellipse_d5_c4_0_3_frequentist.eps}
}
\subfigure[]{
\includegraphics[scale=0.15]{ellipse_d5_c4_0_4_frequentist.eps}
}
\subfigure[]{
\includegraphics[scale=0.15]{ellipse_d5_c4_1_2_frequentist.eps}
}
\subfigure[]{
\includegraphics[scale=0.15]{ellipse_d5_c4_1_3_frequentist.eps}
}
\subfigure[]{
\includegraphics[scale=0.15]{ellipse_d5_c4_1_4_frequentist.eps}
}
\subfigure[]{
\includegraphics[scale=0.15]{ellipse_d5_c4_2_3_frequentist.eps}
}
\subfigure[]{
\includegraphics[scale=0.15]{ellipse_d5_c4_2_4_frequentist.eps}
}
\subfigure[]{
\includegraphics[scale=0.15]{ellipse_d5_c4_3_4_frequentist.eps}
}
\caption{
This plot shows all 10 2-dimensional sub-spaces of our 5-dimensional toy likelihood
function with 4 minima in $\chi^2$.  The black points are $\chi^2\le\chi^2_\text{lim}$
found by APS after 10,000 samplings.  The red contours are the known $\chi^2=\chi^2_\text{lim}$
contours of the function.
}
\label{fig:toyFreq4}
\end{figure*}

\begin{figure*}
\subfigure[]{
\includegraphics[scale=0.15]{ellipse_d5_c2_0_1_bayes.eps}
}
\subfigure[]{
\includegraphics[scale=0.15]{ellipse_d5_c2_0_2_bayes.eps}
}
\subfigure[]{
\includegraphics[scale=0.15]{ellipse_d5_c2_0_3_bayes.eps}
}
\subfigure[]{
\includegraphics[scale=0.15]{ellipse_d5_c2_0_4_bayes.eps}
}
\subfigure[]{
\includegraphics[scale=0.15]{ellipse_d5_c2_1_2_bayes.eps}
}
\subfigure[]{
\includegraphics[scale=0.15]{ellipse_d5_c2_1_3_bayes.eps}
}
\subfigure[]{
\includegraphics[scale=0.15]{ellipse_d5_c2_1_4_bayes.eps}
}
\subfigure[]{
\includegraphics[scale=0.15]{ellipse_d5_c2_2_3_bayes.eps}
}
\subfigure[]{
\includegraphics[scale=0.15]{ellipse_d5_c2_2_4_bayes.eps}
}
\subfigure[]{
\includegraphics[scale=0.15]{ellipse_d5_c2_3_4_bayes.eps}
}
\caption{
This plot shows all 10 2-dimensional sub-spaces of our 5-dimensional toy likelihood
function with 2 minima in $\chi^2$.  The black points represent the $95\%$ Bayesian credible
limit found by APS (and determined as described in Section \ref{sec:bayes}) after 10,000
samplings.  The red contours represent the known $95\%$ Bayesian credible limit of the toy
function.
}
\label{fig:toyBayes2}
\end{figure*}

\begin{figure*}
\subfigure[]{
\includegraphics[scale=0.15]{ellipse_d5_c3_0_1_bayes.eps}
}
\subfigure[]{
\includegraphics[scale=0.15]{ellipse_d5_c3_0_2_bayes.eps}
}
\subfigure[]{
\includegraphics[scale=0.15]{ellipse_d5_c3_0_3_bayes.eps}
}
\subfigure[]{
\includegraphics[scale=0.15]{ellipse_d5_c3_0_4_bayes.eps}
}
\subfigure[]{
\includegraphics[scale=0.15]{ellipse_d5_c3_1_2_bayes.eps}
}
\subfigure[]{
\includegraphics[scale=0.15]{ellipse_d5_c3_1_3_bayes.eps}
}
\subfigure[]{
\includegraphics[scale=0.15]{ellipse_d5_c3_1_4_bayes.eps}
}
\subfigure[]{
\includegraphics[scale=0.15]{ellipse_d5_c3_2_3_bayes.eps}
}
\subfigure[]{
\includegraphics[scale=0.15]{ellipse_d5_c3_2_4_bayes.eps}
}
\subfigure[]{
\includegraphics[scale=0.15]{ellipse_d5_c3_3_4_bayes.eps}
}
\caption{
This plot shows all 10 2-dimensional sub-spaces of our 5-dimensional toy likelihood
function with 3 minima in $\chi^2$.  The black points represent the $95\%$ Bayesian credible
limit found by APS (and determined as described in Section \ref{sec:bayes}) after 10,000
samplings.  The red contours represent the known $95\%$ Bayesian credible limit of the toy
function.
}
\label{fig:toyBayes3}
\end{figure*}

\begin{figure*}
\subfigure[]{
\includegraphics[scale=0.15]{ellipse_d5_c4_0_1_bayes.eps}
}
\subfigure[]{
\includegraphics[scale=0.15]{ellipse_d5_c4_0_2_bayes.eps}
}
\subfigure[]{
\includegraphics[scale=0.15]{ellipse_d5_c4_0_3_bayes.eps}
}
\subfigure[]{
\includegraphics[scale=0.15]{ellipse_d5_c4_0_4_bayes.eps}
}
\subfigure[]{
\includegraphics[scale=0.15]{ellipse_d5_c4_1_2_bayes.eps}
}
\subfigure[]{
\includegraphics[scale=0.15]{ellipse_d5_c4_1_3_bayes.eps}
}
\subfigure[]{
\includegraphics[scale=0.15]{ellipse_d5_c4_1_4_bayes.eps}
}
\subfigure[]{
\includegraphics[scale=0.15]{ellipse_d5_c4_2_3_bayes.eps}
}
\subfigure[]{
\includegraphics[scale=0.15]{ellipse_d5_c4_2_4_bayes.eps}
}
\subfigure[]{
\includegraphics[scale=0.15]{ellipse_d5_c4_3_4_bayes.eps}
}
\caption{
This plot shows all 10 2-dimensional sub-spaces of our 5-dimensional toy likelihood
function with 4 minima in $\chi^2$.  The black points represent the $95\%$ Bayesian credible
limit found by APS (and determined as described in Section \ref{sec:bayes}) after 10,000
samplings.  The red contours represent the known $95\%$ Bayesian credible limit of the toy
function.
}
\label{fig:toyBayes4}
\end{figure*}

\section{A Real Example: WMAP}
\label{sec:wmap}

Section \ref{sec:toy} demonstrates the robustness of APS against multi-modal
$\chi^2$ functions.  In this section, we demonstrate 
that, on well-behaved likelihood functions, APS' rate of convergence is
comparable to that of MCMC.  
We also show that APS can be used on actual astrophysical
problems by using for our $\chi^2$ function the likelihood function on the
7-year data release of the WMAP CMB satellite \cite{wmap7,wmap7likelihood}.
For simplicity, we consider only the anisotropy
power spectrum of the temperature-temperature correlation function.

The parameter space we explore is the six-dimensional space of the spatially
flat concordance $\Lambda$CDM cosmology:
\begin{itemize}
\item$\Omega_bh^2$ -- the density of baryons in the Universe
\\
\item$\Omega_\text{CDM}h^2$ -- the density of cold dark matter in the Universe
\\
\item$\tau$ -- the optical depth to the surface of last scattering.  $\tau$ cannot be
constrained without polarizationd data, which we do not consider here.  However, it functions
as a nuisance parameter to be marginalized over.  The fact that APS can handle it as well as
MCMC is an important test of APS' feasibility.
\\
\item$h$ -- the Hubble parameter in units of
$100~\text{km}~\text{s}^{-1}~\text{Mpc}^{-1}$
\\
\item$n_s$ -- the spectral index controlling the distribution of scalar density
perturbations in the Universe
\\
\item$\ln[10^{10}A_s]$ -- the amplitude of the initial scalar density
perturbations in the Universe
\end{itemize}
These are the parameters taken by the publicly available MCMC code CosmoMC,
which we use to run our baseline MCMC chains \cite{cosmomc}.  The
$\vec{\theta}\rightarrow\chi^2$ function in this case involves converting these
paramters into the power spectrum of anisotropy $C_\ell$s on the sky and
comparing those $C_\ell$s to the actual measurements registered by the WMAP
satellite.  Using the (very fast) Boltzmann code CAMB to calculate the $C_\ell$s
\cite{camb} takes 1.3 seconds on a 2.5 GHz processor.  
Evaluating the $\chi^2$ using the likelihood code provided by the WMAP team
\cite{wmap7likelihood} takes an additional 2 seconds.  Hence the need
for an algorithm like APS or MCMC to accelerate the search.
We will consider both equations (\ref{eqn:GaussianCovar}) and (\ref{eqn:NNCovar})
as covariogram for our Gaussian Process, 
though we find that the choice has little effect on our
results.

To gauge the appropriateness of applying APS to this problem, we consider
the two following tests.  In Figure \ref{fig:chimin} we plot the minimum discovered value
of $\chi^2$ yielded by APS and MCMC as a function of the number of samples drawn
by each.  If APS failed to find a suitable value of $\chi^2_\text{min}$, the use of
$\Delta\chi^2$  to find $\chi^2_\text{lim}$ would be questionable.  However, as we can see,
the $\chi^2_\text{min}$ yielded by APS ($1273.3$) 
is within a few of the $\chi^2_\text{min}$ yielded
by MCMC ($1270.7$).  Our use of $\chi^2_\text{lim}=\chi^2_\text{min}+\Delta\chi^2$ is therefore
justified.

The reader will note that
MCMC discovers its $\chi^2_\text{min}$ much more rapidly than APS, however, 
we will see
in Figures \ref{fig:contours_gauss} through \ref{fig:contours_br} that this rapid
descent to $\chi^2_\text{min}$ does not translate into rapid convergence to
the final credible limit.  Because MCMC infers its limits based on the number
of samples drawn from the Bayesian posterior probability, MCMC still requires
many tens (or hundreds) of thousands of points to be sampled after $\chi^2_\text{min}$
has been discovered.  Because APS is designed to locate points at the precise value
$\chi^2=\chi^2_\text{lim}$, it requires much fewer samples after discovering 
$\chi^2_\text{min}$ to infer the correct confidence limit.

The second question we consider is whether or not our assumption that
$\vec{\theta}\rightarrow\chi^2$ can be modeled by a Gaussian Process is 
valid.  In Figure
\ref{fig:goodness} we plot the average fractional error in $\mu$ as a proxy for $\chi^2$ as a
function of the number of steps sampled (note: the vertical axis actually shows the average
fractional error over the preceding 500 steps so that the early, imprecise models do not
contaminate the later, precise models).  As APS samples more points, its Gaussian Process
model becomes more accurate.  This effect is more pronounced for covariogram (\ref{eqn:GaussianCovar})
than for covariogram (\ref{eqn:NNCovar}).  However, we will see below that this difference
does not effect the parameter constraints determined by APS.

\begin{figure}
\includegraphics[scale=0.3]{chimin_plot.eps}
\caption{
The minimum value of $\chi^2$ found by APS and MCMC on the WMAP 7 problem
as a function of 
steps sampled by each.
}
\label{fig:chimin}
\end{figure}

\begin{figure}
\includegraphics[scale=0.3]{goodness_of_fit.eps}
\caption{
The average value of $|\mu-\chi^2|/\chi^2$ as a function of the number of points
sampled by APS in the WMAP 7 problem.  
This is shown as a check on our assumption that
$\vec{\theta}\rightarrow\chi^2$ can be approximated as a Gaussian Process.
Though the squared exponential covariogram gives a better fit to the $\chi^2$ function,
we see in Figures \ref{fig:contours_gauss} and \ref{fig:contours_nn}
that this has little effect on the parameter constraints derived by APS.
}
\label{fig:goodness}
\end{figure}

We now consider the ultimate results, namely the 95\% confidence limits on
our 6-dimensional cosmological parameter space yielded by APS.
To provide a baseline against which to test APS, we run 4 independent MCMC
chains using CosmoMC.  Like the MCMC code used in Section \ref{sec:toy},
CosmoMC periodically adjusts its proposal density by learning the covariance
matrix of the points it has already sampled.  At each step, it proposes
a new point in parameter space by randomly selecting an eigen vector
of that learned covariance matrix and stepping along it.
We integrate the posterior by taking the output chains, discarding the first
50\% of steps as a burn in period, and thinning the remaining steps so that
we have a set of effectively independent samples drawn from the posterior.
Quantitatively, this is determined by keeping only every $L$th sample after
burn-in, where $L$ is set as the length such that the normalized correlation
\begin{equation}
\frac{\text{Cov}[\theta_n^{(i)},\theta_n^{(i+L)}]}
{\text{Var}[\theta_n]}\le 0.1
\end{equation}
where $n$ is the index over the $N_p$ parameters and $(i)$ is the index
over the samples drawn.

We find the 95\% credible
limits in two-dimensional slices of our parameter space using
a total of 460,000 MCMC steps.  These are assumed to be the
true credible limits on the parameter space and are plotted
as the thick red
contours in Figures \ref{fig:contours_gauss} through 
\ref{fig:contours_br} 
below.  Against these control contours, we plot the Frequentist
confidence limits (all points with $\chi^2\le\chi^2_\text{lim}$)
discovered by APS after only 50,000 evaluations of $\chi^2$.
We consider both an adaptive $\chi^2_\text{lim}=\chi^2_\text{min}+12.6$
and an absoulte $\chi^2_\text{lim}=1283.3$ (this is the 95\% confidence limit
for a $\chi^2$ probability distribution with 1199 degrees of freedom).
These are the black points in Figures \ref{fig:contours_gauss}, \ref{fig:contours_nn},
and \ref{fig:contours_br} respectively.  These confidence limits overlap very nicely
with the credible limits discovered by MCMC after many more $\chi^2$ evaluations,
comporting the the idea that, in the large data limit, Bayesiand credible limits
and Frequentist confidence limits ought to align.

For comparison of convergence properties, we plot the Bayesian credible limits discovered
by MCMC after only 50,000 $\chi^2$ evaluations.  These are the black regions in
Figure \ref{fig:mcmc_comparison}.  As can be seen, MCMC has not yet totally converged
in all of the sub-spaces of cosmological parameter space.  This should be compared
with the reasonably complete coverage of the control credible limits seen in
Figures \ref{fig:contours_gauss}-\ref{fig:contours_br}.

In Figure \ref{fig:contours_bayes}, we plot the 95\% Bayesian credible limit discovered
by APS after 50,000 evalutations of $\chi^2$ and determined by the method outlined in
Section \ref{sec:bayes}.  While we do not see the clean equivalence between the APS-derived credible limits and
the true credible limits that we saw in Figures \ref{fig:toyBayes2}-\ref{fig:toyBayes4},
we do find that the APS-derived credible limits roughly sketch the bounds of the 
true credible limit.  This, combined with its robustness against multi-modal likelihood
functions, suggests that APS may, at the very least be useful in cases where very little
prior information is available and exploring the paramter space is at least as important as
integrating it to find a rigorous bound.

As a final note, we do not plot any confidence or 
credible limits for $\tau$.
This is because $\tau$ needs CMB polarization data for meaningful constraints.  For the sake
of simplicity, we did not consider CMB polarization data here.  Thus, $\tau$ has become a
``nuisance parameter'': it effects the fit of our theoretical models, but cannot itself be
meaningfully constrained.  MCMC handles such parameters by effectively integrating over them
when delivering the two-dimensional contours of Figures
\ref{fig:contours_gauss} through \ref{fig:contours_br}.  
APS handles such nuisance parameters by
simply allowing them to vary and returning any combinations of useful-plus-nuisance parameters
that satisfy the $\chi^2\le\chi^2_\text{lim}$ criterion.  The fact that this nuisance
paramter did not significantly impede the convergence of APS relative to MCMC speaks 
well of
APS' ability to handle actual physical problems which 
often include parameters that have
little physical interest but are required to, e.g., 
calibrate noise distributions or
systematics.

\begin{figure*}
\subfigure[]{
\includegraphics[scale=0.15]{gaussLearn_0_1_50k_frequentist.eps}
}
\subfigure[]{
\includegraphics[scale=0.15]{gaussLearn_0_2_50k_frequentist.eps}
}
\subfigure[]{
\includegraphics[scale=0.15]{gaussLearn_0_4_50k_frequentist.eps}
}
\subfigure[]{
\includegraphics[scale=0.15]{gaussLearn_0_5_50k_frequentist.eps}
}
\subfigure[]{
\includegraphics[scale=0.15]{gaussLearn_1_2_50k_frequentist.eps}
}
\subfigure[]{
\includegraphics[scale=0.15]{gaussLearn_1_4_50k_frequentist.eps}
}
\subfigure[]{
\includegraphics[scale=0.15]{gaussLearn_1_5_50k_frequentist.eps}
}
\subfigure[]{
\includegraphics[scale=0.15]{gaussLearn_2_4_50k_frequentist.eps}
}
\subfigure[]{
\includegraphics[scale=0.15]{gaussLearn_2_5_50k_frequentist.eps}
}
\subfigure[]{
\includegraphics[scale=0.15]{gaussLearn_4_5_50k_frequentist.eps}
}
\caption{
The 10 2-dimensional sub-spaces of our 6-dimensional parameter space
($\tau$ is ignored because not real constraint can be gleaned from this data).
The thick red contours are the 95\% Bayesian credible limits determined by MCMC after
460,000 calls to $\chi^2$.  The black points are the 95\% Frequentist confidence limits
determined by APS after 50,000 calls to $\chi^2$.  $\chi^2_\text{lim}=\chi^2_\text{min}+12.6$.
Equation (\ref{eqn:GaussianCovar}) is used for the Gaussian Process covariogram.
}
\label{fig:contours_gauss}
\end{figure*}

\begin{figure*}
\subfigure[]{
\includegraphics[scale=0.15]{nnLearn_0_1_50k_frequentist.eps}
}
\subfigure[]{
\includegraphics[scale=0.15]{nnLearn_0_2_50k_frequentist.eps}
}
\subfigure[]{
\includegraphics[scale=0.15]{nnLearn_0_4_50k_frequentist.eps}
}
\subfigure[]{
\includegraphics[scale=0.15]{nnLearn_0_5_50k_frequentist.eps}
}
\subfigure[]{
\includegraphics[scale=0.15]{nnLearn_1_2_50k_frequentist.eps}
}
\subfigure[]{
\includegraphics[scale=0.15]{nnLearn_1_4_50k_frequentist.eps}
}
\subfigure[]{
\includegraphics[scale=0.15]{nnLearn_1_5_50k_frequentist.eps}
}
\subfigure[]{
\includegraphics[scale=0.15]{nnLearn_2_4_50k_frequentist.eps}
}
\subfigure[]{
\includegraphics[scale=0.15]{nnLearn_2_5_50k_frequentist.eps}
}
\subfigure[]{
\includegraphics[scale=0.15]{nnLearn_4_5_50k_frequentist.eps}
}
\caption{
The 10 2-dimensional sub-spaces of our 6-dimensional parameter space
($\tau$ is ignored because not real constraint can be gleaned from this data).
The thick red contours are the 95\% Bayesian credible limits determined by MCMC after
460,000 calls to $\chi^2$.  The black points are the 95\% Frequentist confidence limits
determined by APS after 50,000 calls to $\chi^2$.  $\chi^2_\text{lim}=\chi^2_\text{min}+12.6$.
Equation (\ref{eqn:NNCovar}) is used for the Gaussian Process covariogram.
}
\label{fig:contours_nn}
\end{figure*}

\begin{figure*}
\subfigure[]{
\includegraphics[scale=0.15]{gaussAssert_0_1_50k_frequentist.eps}
}
\subfigure[]{
\includegraphics[scale=0.15]{gaussAssert_0_2_50k_frequentist.eps}
}
\subfigure[]{
\includegraphics[scale=0.15]{gaussAssert_0_4_50k_frequentist.eps}
}
\subfigure[]{
\includegraphics[scale=0.15]{gaussAssert_0_5_50k_frequentist.eps}
}
\subfigure[]{
\includegraphics[scale=0.15]{gaussAssert_1_2_50k_frequentist.eps}
}
\subfigure[]{
\includegraphics[scale=0.15]{gaussAssert_1_4_50k_frequentist.eps}
}
\subfigure[]{
\includegraphics[scale=0.15]{gaussAssert_1_5_50k_frequentist.eps}
}
\subfigure[]{
\includegraphics[scale=0.15]{gaussAssert_2_4_50k_frequentist.eps}
}
\subfigure[]{
\includegraphics[scale=0.15]{gaussAssert_2_5_50k_frequentist.eps}
}
\subfigure[]{
\includegraphics[scale=0.15]{gaussAssert_4_5_50k_frequentist.eps}
}
\caption{
The 10 2-dimensional sub-spaces of our 6-dimensional parameter space
($\tau$ is ignored because not real constraint can be gleaned from this data).
The thick red contours are the 95\% Bayesian credible limits determined by MCMC after
460,000 calls to $\chi^2$.  The black points are the 95\% Frequentist confidence limits
determined by APS after 50,000 calls to $\chi^2$.  $\chi^2_\text{lim}=1283.3$.
Equation (\ref{eqn:GaussianCovar}) is used for the Gaussian Process covariogram.
}
\label{fig:contours_br}
\end{figure*}

\begin{figure*}
\subfigure[]{
\includegraphics[scale=0.15]{mcmc_comparison_0_1_50k.eps}
}
\subfigure[]{
\includegraphics[scale=0.15]{mcmc_comparison_0_2_50k.eps}
}
\subfigure[]{
\includegraphics[scale=0.15]{mcmc_comparison_0_4_50k.eps}
}
\subfigure[]{
\includegraphics[scale=0.15]{mcmc_comparison_0_5_50k.eps}
}
\subfigure[]{
\includegraphics[scale=0.15]{mcmc_comparison_1_2_50k.eps}
}
\subfigure[]{
\includegraphics[scale=0.15]{mcmc_comparison_1_4_50k.eps}
}
\subfigure[]{
\includegraphics[scale=0.15]{mcmc_comparison_1_5_50k.eps}
}
\subfigure[]{
\includegraphics[scale=0.15]{mcmc_comparison_2_4_50k.eps}
}
\subfigure[]{
\includegraphics[scale=0.15]{mcmc_comparison_2_5_50k.eps}
}
\subfigure[]{
\includegraphics[scale=0.15]{mcmc_comparison_4_5_50k.eps}
}
\caption{
The 10 2-dimensional sub-spaces of our 6-dimensional parameter space
($\tau$ is ignored because not real constraint can be gleaned from this data).
The thick red contours are the 95\% Bayesian credible limits determined by MCMC after
460,000 calls to $\chi^2$.  The black regions are the 95\% Bayesian credible limits
determined by MCMC after 50,000 calls to $\chi^2$.
}
\label{fig:mcmc_comparison}
\end{figure*}

\begin{figure*}
\subfigure[]{
\includegraphics[scale=0.15]{gaussLearn_0_1_50k_bayes.eps}
}
\subfigure[]{
\includegraphics[scale=0.15]{gaussLearn_0_2_50k_bayes.eps}
}
\subfigure[]{
\includegraphics[scale=0.15]{gaussLearn_0_4_50k_bayes.eps}
}
\subfigure[]{
\includegraphics[scale=0.15]{gaussLearn_0_5_50k_bayes.eps}
}
\subfigure[]{
\includegraphics[scale=0.15]{gaussLearn_1_2_50k_bayes.eps}
}
\subfigure[]{
\includegraphics[scale=0.15]{gaussLearn_1_4_50k_bayes.eps}
}
\subfigure[]{
\includegraphics[scale=0.15]{gaussLearn_1_5_50k_bayes.eps}
}
\subfigure[]{
\includegraphics[scale=0.15]{gaussLearn_2_4_50k_bayes.eps}
}
\subfigure[]{
\includegraphics[scale=0.15]{gaussLearn_2_5_50k_bayes.eps}
}
\subfigure[]{
\includegraphics[scale=0.15]{gaussLearn_4_5_50k_bayes.eps}
}
\caption{
The 10 2-dimensional sub-spaces of our 6-dimensional parameter space
($\tau$ is ignored because not real constraint can be gleaned from this data).
The thick red contours are the 95\% Bayesian credible limits determined by MCMC after
460,000 calls to $\chi^2$.  The black points are the 95\% Bayesian credible limits determined
by APS according to the method outlined in Section \ref{sec:bayes}.  While the APS credible limit
covers the entire MCMC limit, it also covers regions of parameter space excluded by MCMC.
}
\label{fig:contours_bayes}
\end{figure*}

\section{Conclusions}
\label{sec:conclusions}

In Section \ref{sec:toy} we indicated that APS is more robust
than MCMC against multi-modal likelihood functions.
In Section \ref{sec:wmap} we showed that, on more well-behaved
likelihood functions, APS discovers comparable parameter constraints
as MCMC and that APS does not converge any more slowly than does MCMC.
We also showed a great deal of qualitative similarity but quantitative
difference between the Bayesian credible limits drawn by MCMC
and the frequentist confidence limits drawn by APS.  The different
convergence properties and different outputs of APS and MCMC lend themselves
to different use cases.

As was demonstrated in Figure \ref{fig:chimin}, MCMC is significantly
more efficient than APS at finding the true value of $\chi^2_\text{min}$
in a given parameter space.  This makes it ideally suited to problems where
the uncertainty of the data set is poorly understood and must be fit
for.  However, as we saw in Figure \ref{fig:toy}, this rapid convergence to
$\chi^2_\text{min}$ can also result in a myopic disregard for other regions of
interest in parameter space.  Once MCMC has been used to learn the uncertainty
model of the data, it may be advisable to run APS as a guarantor that no
alternative best-fit models have been ignored by MCMC.

In cases where the uncertainty model is well-understood and the contraints in
parameter space can safely be described by either the likelihood ratio's
$\chi^2\le\chi^2_\text{min}+\Delta\chi^2$ or an imposed
$\chi^2\le\chi^2_\text{lim}$, APS' robustness against multi-modality and
comparatively rapid convergence in a confidence limit sense (recall Figures
\ref{fig:contours_gauss} through \ref{fig:contours_br}) make it ideal for rapidly
drawing constraints on parameter space.

One advantage of MCMC that APS cannot overcome is its comparative ease to
implement.  To this end,
we make our code available at \verb|https://github.com/uwssg/APS/|.  
The code is presented as a series of C++ modules with directions
indicating where the user can interface with 
the desired $\vec{\theta}\rightarrow\chi^2$ function.  
Those with questions about the code or the algorithm should not
hesitate to contact the authors.

%%%%%%%%%%%%%%%%%%%%%%%%%%%%%%%%%%%%%%%%%%%%%%%%%%%%%%%%%%%%
%%%%%%%%%%%%%%%%%%%%%%%%%%%%%%%%%%%%%%%%%%%%%%%%%%%%%%%%%%%%

\section*{Acknowledgments}
The authors would like to thank the referee for comments and suggestions
which we hope have made this paper more useful to astrophysical audiences.
They would also like to thank Christopher Genovese for useful discussions
regarding statistial theory and interpretation.
SFD would like to thank Eric Linder, Arman Shafieloo, and Jacob Vanderplas 
for useful conversations
about Gaussian processes.  SFD would also like to acknowledge the hospitality of
the Institute for the Early Universe at Ewha Womans University in 
Seoul, Korea, who
hosted him while some of this work was done.
We acknowledge support from DOE award grant number DESC0002607 and NSF grant
IIS-0844580.

\begin{appendix}
\section{Frequentist Confidence Intervals}
\label{sec:frequentism}

Frequentist confidence limits are not nearly as common in the astrophysical
literature as their Bayesian counterparts.  For this reason, we define them below.
Note, however, that, though we have chosen to use a Frequentist
interpretation for the purposes of this current work, \APS is ultimately agnostic
about the user's chosen statistical interpretation. 

Suppose we are going to conduct some experiment resulting in a random data set
$\vec{d}$.  Suppose also that we have some theory by which the distribution of
$\vec{d}$ is controlled by a set of parameters $\vec{\theta}$.  We would like to
use $\vec{d}$ to make some statement about the true values of $\vec{\theta}^{(0)}$
i.e., the values of $\vec{\theta}$ actually realized by this Universe.
Rather than adopt the Bayesian perspective 
and talk about the peak and width of the
the posterior
``probability of $\vec{\theta}$ given $\vec{d}$,'' Frequentists attempt to
construct sets of estimators
$\vec{\theta}^{(\text{CL})}(\vec{d})$ such that the
probability that 
$\vec{\theta}^{(0)}\in\vec{\theta}^{(\text{CL})}(\vec{d})$
is $(1-\alpha)\%$, i.e. if the experiment were repeated many times, yielding
many different sets $\vec{d}$, and the confidence limit set
$\vec{\theta}^{(\text{CL})}$
was calculated for each of those data sets, the true
value of $\vec{\theta}^{(0)}$ would fall within $\vec{\theta}^{(\text{CL})}$ 
$(1-\alpha)\%$ of the time.
This is the original definition of confidence intervals posited by Neyman (1937;
see his equation 20 and the attendant discussion).  It also underlies the
confidence belt construction discussed 
in Chapter 9 of Eadie {\it et al}. (1971), Chapter 20 of Stuart and Ord (1991)
and section II B of Feldman and Cousins (1998).

The specific example we consider in Section \ref{sec:wmap}
of this paper is the WMAP 7 year data
release measuring the anisotropy in the Cosmic Microwave Background.  This 
data set takes the form of a
power spectrum composed of 1199 $C_\ell$s ($2\le\ell\le1200$) 
characterizing the
anisotropy power on different angular scales.  
These 1199 measurements represent
independent Gaussian random variables. 
The probability density on $C_\ell$ space associated with
measuring a particular set of $C_\ell$s given a particular candidate set of
$\vec{\theta}^{(c)}$ is
$$P\propto\exp
\big[-\frac{1}{2}
\sum_{\ell,\ell^\prime}(C_\ell(\vec{d})-C_\ell(\vec{\theta}^{(c)})) 
\text{Cov}^{-1}_{\ell\ell^\prime}
(C_{\ell^\prime}(\vec{d})-C_{\ell^\prime}(\vec{\theta}^{(c)}))\big]$$
where 
$C_\ell(\vec{d})$ are the 1199 measured values of $C_\ell$,
$C_\ell(\vec{\theta}^{(c)})$ are the 1199 values of $C_\ell$ predicted
by the candidate theory,
and $\text{Cov}^{-1}_{\ell\ell^\prime}$ is the inverse of the covariance matrix
relating the measured $C_\ell$s.
A change of variables to
$$\chi^2\equiv \sum_{\ell\ell^\prime}
(C_\ell(\vec{d})-\vec{C}_\ell(\vec{\theta}^{(c)}))
\text{Cov}^{-1}_{\ell\ell^\prime}
(C_{\ell^\prime}(\vec{d})-C_{\ell^\prime}(\vec{\theta}^{(c)}))$$
gives the well-known result that the 1199 independent, Gaussian-distributed
$C_\ell$s give a $\chi^2$ statistic distributed according to the eponymous
$\chi^2$ distribution with 1199 degrees of freedom.  The $\chi^2$ distribution
is well-understood.  In the case of 1199 degrees of freedom, 95\% of the
probability is enclosed by $\chi^2\le1281$.  
We may use this fact to construct a
95\% confidence limit.

Recall the definition of Frequentist confidence limits.  The true
value of the parameters
$\vec{\theta}^{(0)}$ is fixed, but $\vec{d}$ (in the specific example above, the
set of 1199 $C_\ell$s) is random and will change each time we conduct the
experiment.  In this case, ``conducting the experiment'' means creating a new Universe
with a new Cosmic Microwave Background, randomly generated from the same
$\vec{\theta}^{(0)}$, and measuring it with WMAP;
while this is impossible, we ask the readers to
suspend their disbelief for the sake of this illustration.  
Because of what we have just noted about $\chi^2$, in
95\% of our repeated WMAP experiments $\chi^2$ calculated relative to the true
parameter set $\vec{\theta}^{(0)}$ will be less than or equal to 1281.  If, for
each set of $C_\ell$s we find all of the combinations of $\vec{\theta}$ that
yield $\chi^2\le1281$, that set of $\vec{\theta}$ will contain $\vec{\theta}^{(0)}$
95\% of the time.
Taking our one realization of $C_\ell$ and finding all of the values of
$\vec{\theta}^{(\text{CL})}$ which give $\chi^2\le1281$,
we can be confident that we have contained $\vec{\theta}^{(0)}$ with 95\%
probability.  This is what is meant by a 95\% Frequentist confidence limit.

In the large data limit, Frequentist confidence limits are expected to
give comparable results to Bayesian inference (as we see in Figure
\ref{fig:contours_gauss}).  In the limit of small data, Frequentist confidence limits
may differ from Bayesian limits.  If statistical fluctuations dominate the
signal, it is possible for Frequentist methods to return an empty confidence
limit (no points in parameter space fit the data).  While this may seem an
unpalatable outcome, it is useful to know that one's data set is noisy so that
one can interpret the explored likelihood surface with all due caution.
Lyons (2002 and 2008) discusses this distinction, as well as the
comparative strengths and weaknesses of other statistical perspectives, at
greater length than this work.

The Frequentist approach to confidence limits described herein
is actually a conservative version of
the non-parametric confidence ball approach popular among academic statisticians
(see the introduction to Baraud 2004).  Statisticians typically are concerned
with using data to constrain the form of a function in arbitrary function space.
They use the data to derive an estimator of the function 
from which it was drawn (in our example, the smooth theoretical $C_\ell$ function)
and then use
theoretical considerations to draw a hypersphere in function space about that
fit constituting the $(1-\alpha)\%$ confidence limit
\cite{bd,li,baraud,cai,davies}.
Note that these hyperspheres (referred to in the literature as ``confidence
balls'') are calculated from the data alone without reference to any
underlying physical model or set of (in Wilks' 1938 words) ``admissible
hypotheses.''  In this way, they function identically to the $\chi^2$ test
adopted in the present work.  

Astrophysicists can utilize the
``confidence ball'' formalism by demanding that their theoretical
models give predicted functions that fall within 
the hypersphere drawn in function space.  This is how
Bryan {\it et al}. (2007) originally drew their constraints on the CMB.  It is
also how Genovese {\it et al}. (2004) confirmed the statistical significance of
the harmonic peaks in the WMAP 1-year CMB data.  Because the confidence
hypersphere is drawn a-priori, without any consideration for the admissible
hypotheses within the space of physical parameters, this method would be
straightforward to interface with \APS (indeed, it was included in the Bryan
{\it et al}. draft of the code).  
We leave such an implementation to future work.

The more familiar likelihood ratio test detailed
by Wilks (1938)  and Neyman and Pearson (1933) is an asymptotic simplification
of Frequentist confidence limits
for cases in which we know that the true model must exist
within some limited set of hypotheses.  
In this case, one assumes that the adopted $\vec{\theta}$ parametrization
is the only possible way of explaining that data.  In that case, the
$\chi^2_\text{min}$ on parameter space must, by definition, be the smallest
achievable $\chi^2$.  $\Delta\chi^2\equiv\chi^2-\chi^2_\text{min}$
is then distributed according to the $\chi^2$ distribution with $N_p$ 
degrees of freedom.  This is the assumption underlying our use of
$\Delta\chi^2$ in the test of Section \ref{sec:toy} and \ref{sec:wmap} above.



\end{appendix}

%%%%%%%%%%%%%%%%%%%%%%%%%%%%%%%%%%%%%%%%%%%%%%%%%%%%%%%%%%%%
%%%%%%%%%%%%%%%%%%%%%%%%%%%%%%%%%%%%%%%%%%%%%%%%%%%%%%%%%%%%
\begin{thebibliography}{99}

\bibitem[Atchad\'e and Rosenthal 2005]{mcmcefficiency}
Atchad\'e, Y.~F. and Rosenthal, J.~S., 2005, Bernoulli {\bf 11}, 815

\bibitem[Baraud 2004]{baraud}
Baraud,~Y. 2004, Annals of Statistics {\bf 32} 528

\bibitem[Bentley 1975]{kdtree}
Bentley,~J.~L. 1974, Communications of the Associaton for Computing Machinery
{\bf 18}, 509

\bibitem[Beran and D\"umbgen 1998]{bd}
Beran,~R. and D\"umben,~L. 1998, Annals of Statistics {\bf 26} 1826

\bibitem[Berg\'e {\it et al}. 2012]{psf}
Berg\'e,~J., Price,~S., Amara,~A., and Rhodes,~J. 2012,
Monthly Notices of the Royal Astronomical Societ {\bf 419}, 2356

\bibitem[Brooks and Gelman 1998]{mcmcconvergence}
Brooks, S. and Gelman, A., 1998, 
Journal of Computational and Graphical Statistics
{\bf 7} 434

\bibitem[Bryan 2007]{brentsthesis}
Bryan, B., 2007, Ph.D. thesis
\verb|http://reports-archive.adm.cs.cmu.edu/anon/|
\verb|ml2007/abstracts/07-122.html|

\bibitem[Bryan {\it et al}. 2007]{bryan}
Bryan, B., Schneider, J., Miller, C.~J., Nichol, R.~C., Genovese, C., and
Wasserman, L., 2007,
Astrophys. \ J. {\bf 665}, 25

\bibitem[Cai and Low 2006]{cai}
Cai,~T.~T. and Low,~M.~G. 2006, Annals of Statistics {\bf 34} 202

\bibitem[Davies {\it et al}. 2009]{davies}
Davies,~P.~L., Kovac,~A., and Meise,~M. 2009, Annals of Statistics,
{\bf 37} 2597

\bibitem[Dunkley {\it et al}. 2005]{mcmcdunkley}
Dunkley, J. Bucher, M., Ferreira, P.~G., Moodley, K., and Skordis, C.,
2005,  
Mon. \ Not. \ R. \ Astron. \ Soc. {\bf 356},
925

\bibitem[Eadie {\it et al}. 1971]{eadie}
Eadie,~W.~T., Drijard,~D., James,~F.~E., Roos,~M., and Sadoulet,~B.,
1971, ``Statistical Methods in Experimental Physics''
(North-Holland, Amsterdam).

\bibitem[Feldman and Cousins 1998]{feldman}
Feldman,~G.~J. and Cousins,~R.~D. 1998, Phys. Rev. D. {\bf 57} 3873

\bibitem[Gelman {\it et al}. 1996]{gelman}
Gelman, A., Roberts, G. O., and Gilks, W. R. 1996,
Bayesian Statistics {\bf 5}, 599

\bibitem[Genovese {\it et al}. 2004]{genovese}
Genovese,~C.~R., Miller,~C.~J., Nichol,~R.~C., Arjunwadkar,~M., and
Wasserman,~L. 2004, Statistical Science {\bf 19} 308

\bibitem[Gilks {\it et al}. 1996]{mcmc}
Gilks, W.~R., Richardson, S., and Spiegelhalter, D.~J. ed.,
``Markov Chain Monte Carlo in Practice,'' (Chapman \& Hall, 1996),
Boca Raton, F.L.

\bibitem[Habib {\it et al}. 2007]{habib}
Habib,~S., Heitmann,~K., Higdon,~D., Nakhleh,~C., and
Williams,~B. 2007, Physical Review D {\bf 76} 083503

\bibitem[James 1972]{minuit}
James,~F. 1972, Proceedings of the CERN Computing and Data Processing School,
\verb|http://seal.web.cern.ch/seal/documents/minuit.mntutorial.pdf|

\bibitem[Jarosik {\it et al.} 2011]{wmap7}
N.~Jarosik {\it et al}., 2011, Astrophys. \ J. \ Suppl. \ Ser. {\bf 192} 14

\bibitem[Li 1989]{li}
Li,~K.-C. 1989, Annals of Statistics {\bf 17} 1001

\bibitem[Lewis and Bridle 2002]{cosmomc}
Lewis, A. and Bridle, S., 2002, Phys.\ Rev. \ D {\bf 66}, 103511;
\verb|http://cosmologist.info/cosmomc/readme.html|

\bibitem[Lewis {\it et al}. 2000]{camb}
Lewis, A., Challinor, A., and Lasenby, A., 2000, 
Astrophys. \ J. {\bf 538}, 473

\bibitem[Lyons 2002]{lyonsbf}
Lyons,~L. 2002, ``Bayes or Frequentism?''
\verb|http://www-cdf.fnal.gov/physics/statistics/|
\verb|statistics_recommendations.html|

\bibitem[Lyons 2008]{lyonsall}
Lyons,~L. 2008, Annals of Applied Statistics, {\bf 2} 887
[arXiv:0811.1663]

\bibitem[Marsaglia 2003]{random}
Marsaglia,~G. 2003, Journal of Statistical Software, {\bf 8} 1

\bibitem[Nelder and Mead 1965]{simplex}
Nelder,~J.~A. and Mead,~R. 1965, The Computer Journal, {\bf 7} 308

\bibitem[Neyman and Pearson 1933]{np}
Neyman,~J. and Pearson,~E.~S. 1933, Philosophical Transactions
of the Royal Society of London, Series A, {\bf 231} 289

\bibitem[Neyman 1937]{neyman}
Neyman,~J. 1937, Philos. Trans. R. Soc. London {\bf A236} 333.
Reprinted in ``A Selection of Early Statistical Papers of J. Neyman''
(University of California Press, Berkeley, 1967)

\bibitem[Press {\it et al}. 2007]{nr}
Press,~W.~H., Teukolsky,~S.~A., Vetterling,~W.~T. and Flannery,~B.~P. 2007,
``Numerical Recipes'' (3d edition), Cambridge University Press, 2007

\bibitem[Rasmussen and Williams 2006]{gp}
Rasmussen, C.~E. and Williams, C.~K.~I., 2006, ``Processes for Machine Learning''
\verb|http://www.GaussianProcess.org/gpml|

\bibitem[Shafieloo {\it et al}. 2012]{ericgp}
Shafieloo,~A., Kim,~A.~G., and Linger,~E.~V. 2012,
Physical Review D {\bf 85}, 123530 [arXiv:1204.2272]

\bibitem[Stuart and Ord 1991]{stuart}
Stuart,~A. and Ord,~J.~K., 1991, ``Classical Inference and Relationship,''
5th ed., Kendall's Advanced Theory of Statistics, Vol. 2
(Oxford University Press, New York).

\bibitem[Vanderplas and Connolly 2012]{gradientdescent}
Vanderplas, J. and Connolly, A., 2012, [in preparation]

\bibitem[Walsh 2004]{walsh}
Walsh,~B. 2004, Lecture Notes for EEB 581,
\verb|http://www.web.mit.edu/~wingated/www/|
\verb|introductions/mcmc-gibbs-intro.pdf|

\bibitem[WMAP 2010]{wmap7likelihood}
WMAP, 2010
\verb|http://lambda.gsfc.nasa.gov/product/map/|
\verb|current/likelihood_info.cfm|

\bibitem[Wilks 1938]{wilks}
Wilks,~S.~S. 1938, Annals of Mathematical Statistics, {\bf 9} 60

\end{thebibliography} 

\label{lastpage}

\end{document}

%%%%%%%%%%%%%%%%%%%%%%%%%%%%%%%%%%%%%%%%%%%%%%%%%%%%%%%%%%%%
